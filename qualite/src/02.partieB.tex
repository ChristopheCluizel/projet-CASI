\section{B1}
\label{sec:B1}

Plusieurs approches techniques existent dans le domaine du streaming de données. Nous allons travailler avec des tweets mais il est possible d'appliquer toutes ces techniques dans d'autres domaines.\\

\subsection{Envoi des données pour analyse}
\label{sub:Envoi des données pour analyse}

  Tout d'abord, la première solution technique existante est l'envoi des données au fur et à mesure de leur arrivée. Dans notre cas : Twitter envoie chaque tweets au fur et à mesure de leur émission sur ce que l'on appelle un \textit{endpoint}. L'envoi de chaque tweet peut-être effectué suivant plusieurs protocoles différents comme HTTP pour le plus connu ou encore par un protocole créé spécialement pour l'application via de simples sockets réseaux.\\

  \paragraph{Avantage de la solution}
  \label{par:Avantage de la solution}
  Le principal avantage de cette technique est la rapidité du transfert de l'information. Les tweets sont traité à la même vitesse que leur création et l'information créé est donc réellement en temps réel (sans compter les temps de transfert de l'information de Twitter à notre programme de traitement ni le temps de traitement).

  \paragraph{Inconvénients de la solution}
  \label{par:Inconvénients de la solution}
  Cet avantage peut se transformer en inconvénient si l'\textit{endpoint} n'est pas capable d'absorber l'information assez rapidement. Au vu de la quantité de tweets envoyés sur Twitter, il est très compliqué de concevoir un système capable de réagir à la masse de données reçues. Dans le cas où le système ne permet plus de traiter les tweets, il cesse de répondre et les tweets envoyés par Twitter sont donc perdus.\\

  \paragraph{Outils pratiques}
  \label{par:Outils pratiques}
  Twitter ne propose pas actuellement de sytème d'\textit{endpoint}. Ce choix est facilement compréhensible au vu des inconvénients évoqués dans le paragraphe précédent. Il est par contre possible de mettre en place ce genre d'architecture pour des volumes de données moindres comme par exemple lors de l'exécution de l'envoi d'un message sur \href{https://slack.com/}{Slack} ou IRC (plateformes de messagerie instantanée). L'\textit{endpoint} peut dans ce cas exécuter des programmes spécifiques en fonction du message reçu.

\subsection{Utilisation de queues de messages}
\label{sub:Utilisation de queues de messages}

  Afin de résoudre le problème de la masse de tweets reçue par l'\textit{endpoint}, il est possible d'utiliser des queues de messages. Ces queues de messages n'ont pas vocations à traiter les données mais seulement les stocker en attendant la consommation.\\

  Avec cette solution technique, l'action de traitement ne reçoit plus les tweets en temps réel mais peut venir les récupérer dès qu'elle en a besoin dans la queue de messages. Nous ne sommes plus dans une situation de reception mais bien de récupération de l'information.\\

  \paragraph{Avantages de la solution}
  \label{par:Avantages de la solution}
  Cette solution permet de mieux réguler le flux de tweets car la queue n'a plus de problèmes pour absorber la masse de tweets envoyés car elle ne traite pas l'information. Le programme de traitement quant-à-lui peut récupérer l'information dès qu'il en a besoin. Le traitement est donc mieux réparti entre les périodes avec un fort volume de tweets (ou le traitement va prendre du retard) et les périodes de faible volume de tweets (ou le traitement va prendre de l'avance).

  \paragraph{Inconvénients de la solution}
  \label{par:Inconvénients de la solution}
  L'utilisation de la queue est tout de même limitée en terme d'espace de stockage. En effet, si le retard accumulé par le traitement devient trop important et qu'il devient impossible de stocker les tweets dans la queue (par manque de RAM pour une queue en mémoire vive ou d'espace disque pour une queue persistée sur disque), il est nécessaire de supprimer des anciens tweets ou de refuser des nouveaux tweets. Une information est donc perdue.

  \paragraph{Outils pratiques}
  \label{par:Outils pratiques}
  De nombreux outils de queues de messages existent. Il est possible d'utiliser Apache Flume ou Apache Kafka en auto-hébergé. Il est également possible de profiter de l'offre cloud d'AWS nommé Amazon Kinesis. Ces trois outils fonctionnent de la même manière : une source de données (Twitter dans notre cas), un tuyau stockant les tweets et un puit (notre programme de traitement).\\

  Ces trois outils sont des queues de messages distribuées. Cela signifie que ces programmes peuvent fonctionner sur plusieurs machines physiques ou virtuelles. Concernant Amazon Kinesis, l'architecture fonctionne par \textit{shards} qui sont des tuyaux distribués. Il faut donc gérer plusieurs consommateurs de la queue différents. Ces consommateurs ne vont pas accèder à toute l'information mais seulement une partie des tweets. Dans notre cas, les tweets sont des évènements totalement indépendants cette multitude de consommateurs est donc sans importance. Dans le cas où les évènements mis en queue pourrait être liés entre eux, deux solutions sont possibles : aggréger les messages avant l'insertion dans la queue ou choisir une clé de partitionnement permettant d'insérer les messages liés dans le même tuyau et ainsi les traiter avec le même consommateur. 
