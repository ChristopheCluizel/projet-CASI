\section{B1}
\label{sec:B1}

Plusieurs approches techniques existent dans le domaine du streaming de données. Nous allons travailler avec des tweets mais il est possible d'appliquer toutes ces techniques dans d'autres domaines.\\

\subsection{Envoi des données pour analyse}
\label{sub:Envoi des données pour analyse}

  Tout d'abord, la première solution technique existante est l'envoi des données au fur et à mesure de leur arrivée. Dans notre cas : Twitter envoie chaque tweets au fur et à mesure de leur émission sur ce que l'on appelle un \textit{endpoint}. L'envoi de chaque tweet peut-être effectué suivant plusieurs protocoles différents comme HTTP pour le plus connu ou encore par un protocole créé spécialement pour l'application via de simples sockets réseaux.\\

  Le principal avantage de cette technique est la rapidité du transfert de l'information. Cet avantage peut se transformer en inconvénient si l'\textit{endpoint} n'est pas capable d'absorber l'information assez rapidement. Au vu de la quantité de tweets envoyés sur Twitter, il est très compliqué de concevoir un système capable de réagir à la masse de données reçues.\\

\subsection{Utilisation de queues de messages}
\label{sub:Utilisation de queues de messages}

  Afin de résoudre le problème de la masse de tweets reçue par l'\textit{endpoint}, il est possible d'utiliser des queues de messages. Ces queues de messages n'ont pas vocations à traiter les données mais seulement les stocker en attendant la consommation.\\

  Avec cette solution technique, l'action de traitement ne reçoit plus les tweets en temps réel mais peut venir les récupérer dès qu'elle en a besoin dans la queue de messages. Nous ne sommes plus dans une situation de reception mais bien de récupération de l'information.\\
